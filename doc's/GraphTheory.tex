\documentclass[11pt]{article}

\usepackage{sectsty}
\usepackage{graphicx}


% Margins
\topmargin=-0.45in
\evensidemargin=0in
\oddsidemargin=0in
\textwidth=6.5in
\textheight=9.0in
\headsep=0.25in



\title{\underline{\textbf{ Graph Theory Project Research and Documentation}}}
\author{\textbf{Aaron Moran }}
\date{\today}
\renewcommand{\familydefault}{\rmdefault}


\begin{document}
\maketitle	
\pagebreak

% Optional TOC
% \pagebreak


%--Paper--

\section{Project Overview}
Our project involves writing a program in Python to execute regular expressions on
strings using an algorithm known as Thompson’s construction.
You will be required to
demonstrate your program to the lecturer towards the end of the semester.
Please note that all students are bound by the Quality Assurance Framework at GMIT which includes the Code of Student Conduct and the
Policy on Plagiarism. The onus is on the student to ensure they do not, even
inadvertently, break the rules. A clean and comprehensive git history
 is the best way to demonstrate that your submission is your own
work. It is, however, expected that you draw on works that are not your
own and you should systematically reference those works to enhance your
submission. 
\newline
\emph{(Lecturer Ian McLoughlin - Graph Theory Project Overview)}

%--/Paper--
\newpage


\section{Thompsons Construction}

The first theory we have learned within  our module of Graph Theory was Thompsons Construction.
This is the method of converting a regular expression into a NFA - Non-deterministic Finitite Automata.

Within the conversion of a NFA to DFA we see a lot of Epsilon transitions. An Epsilon Transition allows an automaton to change its state when the process is taking place.

%--/Paper--
\newpage

\section{References}

\begin{enumerate}
\item https://learnonline.gmit.ie/course/view.php?id=1599
\item Lecturer Ian McLoughlin - https://github.com/ianmcloughlin
\end{enumerate}



\end{document}